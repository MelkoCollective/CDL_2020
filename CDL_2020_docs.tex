\documentclass[12pt]{article}

% packages
\usepackage{setspace}
\usepackage{array}
\usepackage[margin=0.75in]{geometry}
\usepackage{amsmath,bm}
\usepackage{amssymb}
\usepackage{bbold}
\usepackage{physics}
\usepackage{xcolor}
\usepackage{indentfirst}
\usepackage{enumerate}
\usepackage{mathtools}
\usepackage{fancyhdr}

\pagestyle{fancy}
\fancyhf{}
\rhead{Creative Destruction Lab}
\lhead{Introductions to Projects}
\rfoot{Page \thepage}

\allowdisplaybreaks

\title{CDL Quantum 2020 Cohort Project}

\begin{document}

\maketitle

\thispagestyle{empty}
\section*{Introduction and Objectives}

The 2020 Cohort Project is an online collaboration bringing together founders in a series of open-source challenges.
During the four-week bootcamp, founders will form teams to compete and collaborate on a set of challenges on a general topic or theme
relevant for the Quantum Incubator Stream.  Four separate challenges will be issued; one per week.  New teams will be formed by the Academic Director weekly.

Challenges consist of scientific, computational, and business tasks, devised to be tacked in diverse teams of founders with complementary skill sets.  Each year
the general theme of the Cohort Project will change.  In 2020 it is {\it quantum chemistry}.


\section*{Quantum Chemsitry}

The field of quantum chemistry is a multi-disciplinary effort spanning physics, chemistry, computer science, software engineering, and high-performance computing,
all working together to assist in the design and discovery of new molecules, materials, and drugs.  It is a {\it native quantum} field, meaning that the underlying
foundation of its technology is quantum mechanics. It offers a rich opportunity for progress in classical software, machine learning, quantum-inspired solutions,
and quantum computing.

The behavior of protons and electrons in atoms is governed by the Schrodinger equation $i \hbar \frac{\partial }{ \partial t}  | \Psi \rangle = H | \Psi \rangle$.
Quite generally, this equation cannot be solved exactly for more than a few particles; in order to make progress, a series of well-controlled approximations is 
typically applied to make the equations tractable.  For example, the Born-Oppenheimer approximation assumes that the nuclei do not move due to their 
greater mass than the electrons.  

In this cohort project we will consider diatomic molecules composed of two bonded atoms.
 
\subsection*{Electronic Structure}

Electronic structure is the state of electrons in a static field created by heavy, stationary nuclei.
Except for simple few-body atoms and molecules (like Hydrogen), the solution to electronic structure problems can require significant computational resources.

\subsection*{Rovibrational }

In addition to its electronic structure, the total molecular state depends on its vibrational and rotational wavefunctions.

%\newpage
%\thispagestyle{empty}
%\tableofcontents

\newpage

\section{Weekly Challenges}

\subsection{Week 1: Machine Learning}

\subsection{Week 2: Variational Quantum Eigensolving}

\subsection{Week 3: Franck Condon Factors}

\subsection{Week 4: Ising Annealer}

\newpage

\bibliography{refs}

\end{document}